\documentclass[a4paper]{article}
\usepackage{array}  
\usepackage[table]{xcolor}% http://ctan.org/pkg/xcolor
\usepackage{geometry}
\geometry{margin=1.25in}
\usepackage{hhline}
\usepackage{environ}
 %\geometry{
 %a4paper,z
 %total={170mm,257mm},
 %left=40mm,
 %right=40mm
 %}
 \newcommand{\colWidth}{141mm}

\begin{document} 
\section*{Demo day: \textit{2} Group \textit{18}}

% ------------GOALS----------

\begin{center}
\begin{tabular}{|p{\colWidth}|}
	\hline
	\cellcolor{blue!25}\large
	\textbf{What were your goals?}
	\\ \hline

	\begin{itemize}
		\item Web server
			\begin{itemize}
				\item Endpoint to register a robot to the users account
				\item Only return the list of robots registered to the user's account
				\item Endpoint to send calibration parameters including a range, min and max values
				\item Endpoint to receive calibration parameters
				\item Store current temperature settings
				\item Endpoint to get the current temperature
				\item Endpoint to update the current temperature
			\end{itemize}
		\item Android
			\begin{itemize}
				\item UI for registering a new robot
				\item Dynamically build UI for calibrating the robot
				\item Send calibration parameters back to the server
				\item UI to choose the desired temperature
				\item Send the desired temperature to the server
			\end{itemize}
		\item Robot
			\begin{itemize}
				\item Assign each robot a hard-coded ID
				\item Construct the LEGO dial grip
			\end{itemize}
	\end{itemize} \\

	\hline
\end{tabular}
\vskip 5mm

% ------------ORGANISATION----------

\begin{tabular}{|p{\colWidth}|}
	\hline
	\cellcolor{blue!25}\large
	\textbf{Summarise how your group organised the workload to achieve your goals.}
	\\ \hline
	
		We achieved our goals by distributing them across the team in the following manner:
		\begin{itemize}
			\item Web Server: Ben, Gwion
			\item Android: Theo
			\item Robot:
			\begin{itemize}
				\item ESP8266 development: Spencer, Sameer
				\item Connecting the ESP8266 to Eduroam: Xiaobin, Spencer
				\item Creating the LEGO dial grip: Guanghui, Xiaobin, Luke
				\item 3D-printing (for demo 3): Luke
			\end{itemize}
		\end{itemize}

		We have continued using the \textit{Projects} feature on GitHub to track everyone's progress, along with communicating on Slack and in our weekly meetings.
		This came in extra handy for this demo due to the Creative Learning Week, which meant that some people were either away from Edinburgh or busy with other responsibilities.
	
  \\
  \hline
\end{tabular}
\vskip 5mm

% ------------ACHIEVEMENTS----------

\begin{tabular}{|p{\colWidth}|}
	\hline
	\cellcolor{blue!25}\large
	\textbf{What were your main achievements?}
	\\ \hline
	\vtop to 95mm{
		For this demo we achieved all of the goals listed in the section above. Bringing all of these together, we have
		managed to support robot registration and calibration from end-to-end, along with supporting the thermostat use case.

		\vspace{2mm}

		We also managed to get a head start on some of our goals for demo 3:
		\begin{itemize}
			\item Alexa: Xioabin has researched how to set up the Alexa skill, and Ben, Gwion, and Theo put a lot of work towards switching to OAuth for authorization, which Alexa requires.
			\item 3D-printing: Luke managed to make really good progress on 3D-printing the switch robot, which makes for a much neater product.
		\end{itemize}
	}
  \\
  \hline
\end{tabular}
\vskip 5mm

% ------------NOT ACHIEVED----------

\begin{tabular}{|p{\colWidth}|}
	\hline
	\cellcolor{blue!25}\large
	\textbf{What did you not achieve? Briefly explain why.}
	\\ \hline
	\vtop to 95mm{
		We are proud to say that we achieved all of our goals for this demo.
		We would therefore like to instead discuss the challenges we've faced thus far on the Alexa integration and the 3d-printed switch robot, which are goals for demo 3.

		\vspace{2mm}

		The main issue with Alexa is that it requires an authorization service compatible with OAuth, which is a standard for securing access to APIs.
		Over the last week Ben and Gwion made excellent progress on switching our authorization service over so that it is compatible with the standard,
		but when we were trying to bring it all together with the app we kept running into integration issues. OAuth is a large, quite complex standard
		with a complicated flow, and we do not have any previous experience working with it.

		\vspace{2mm}

		The 3D-printing has proven challenging largely because it takes a lot of time to iterate designs, having to wait for them to print before trying them out and modifying as necessary.
		It also appears that the servos we are using are not strong enough for our current switch-robot design, as it is unable to push the buttons fully in.
		We will now investigate alternative designs closer to that of the LEGO robot to showcased last demo in order to circumvent this issue.
		Alternatively, if we find that we are unable to come up with a better design we may have to purchase more powerful servos.
  }
  \\
  \hline
\end{tabular}
\vskip 5mm

% ------------QUANTITIVE----------

\begin{tabular}{|p{\colWidth}|}
	\hline
	\cellcolor{blue!25}\large
	\textbf{Include any quantitative data you have collected (this can be a graph/table with a few words)}
	\\ \hline
	\vtop to 135mm{
		Power draw analysis:

		\begin{tabular}{ | c | c | }
			\hline
			Mean NodeMCU powerdraw when connected to an AP & 61.5mA \\
			Mean NodeMCU powerdraw when not connected to an AP &  93mA \\
			Mean NodeMCU powerdraw when in deep sleep mode & 0.1mA \\
			\hline
		\end{tabular}

		Using a 3000mAh battery, we estimate that when using the board in WiFi mode it would last about a day.
		If we manage to get the board to be in deep sleep 80\% of the time, this increases to ~15 days.
		We also plan on switching to a more minimal ESP setup, which will draw less current than the NodeMCU board.

		\vspace{2mm}
		
		Comparison of average server response times, tested over 1hr with JMeter, from last demo and this one:

		\begin{tabular}{ | c | c | c |}
			\hline
			Task & This Demo & Last Demo \\
			\hline
			Login request & 653ms & 1992ms \\
			Listing all robots & 259ms & 48ms \\
			Getting one robot & 254ms & 50ms \\
			Calibration request & 251ms & N/A \\
			\hline
		\end{tabular}

		The faster login response time can be attributed to having upgraded the server RAM and switching to gRPC for intra-server message passing.
		The slower response times for the other tasks is due to the fact that they now require authenticating the user.

  }
  \\
  \hline
\end{tabular}
\vskip 5mm

% ------------NEXT STEPS----------

\begin{tabular}{|p{\colWidth}|}
	\hline
	\cellcolor{blue!25}\large
	\textbf{Say briefly what changes you will make to your plan for the next demo.}
	\\ \hline
	\vtop to 45mm{
		We are concerned that OAuth may prove too difficult to implement as there are a lot of nuances to it.
		If we are unable to get this to work then we will not be able to integrate with Alexa, which is one of
		our main goals for the next demo. We will therefore make this our number one priority during the first
		week of development, however if we still haven't gotten it to work by then we will abandon Alexa and
		consider alternative smart-home systems to integrate with.

		\vspace{2mm}

		We have also realized that we need to make room in our plan for addressing the backlog of issues we have
		open on GitHub. These are mostly minor things, such as implementing better logging and adding icons to the display of robots,
		but there are also larger concerns such as making the app pretty (now that we have a large portion of the functionality down)
		which we need to address.
  	}
  \\
  \hline
\end{tabular}

\end{center}
  
\end{document}