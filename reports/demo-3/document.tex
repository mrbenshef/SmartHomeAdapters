\documentclass[a4paper]{article}
\usepackage{array}  
\usepackage[table]{xcolor}% http://ctan.org/pkg/xcolor
\usepackage{geometry}
\geometry{margin=1.25in}
\usepackage{hhline}
\usepackage{environ}
 %\geometry{
 %a4paper,z
 %total={170mm,257mm},
 %left=40mm,
 %right=40mm
 %}
 \newcommand{\colWidth}{141mm}

\begin{document} 
\section*{Demo day: \textit{3} Group \textit{18}}

% ------------GOALS----------

\begin{center}
\begin{tabular}{|p{\colWidth}|}
	\hline
	\cellcolor{blue!25}\large
	\textbf{What were your goals?}
	\\ \hline
	
	\begin{itemize}
		\item Web server 
		\begin{itemize}
			\item Create Alexa skills schema
			\item Add Alexa endpoints for receiving Alexa requests
			\item Endpoint for unlocking/locking bolt lock
			\item Become an OAuth provider
		\end{itemize}
		\item Android
		\begin{itemize}
			\item UI for unlocking/locking bolt lock
			\item Send request to lock/unlock bolt lock
			\item Use OAuth for authentication
			\item Develop and test a mock-up for a better app UI
		\end{itemize}
		\item Robot
		\begin{itemize}
			\item 3D print light switch grip
			\item Build grip to manipulate bolt lock
			\item Add buttons which allow for manual control of the adapters
		\end{itemize}
	\end{itemize}
	\\ \hline
\end{tabular}
\vskip 5mm

% ------------ORGANISATION----------

\begin{tabular}{|p{\colWidth}|}
	\hline
	\cellcolor{blue!25}\large
	\textbf{Summarise how your group organised the workload to achieve your goals.}
	\\ \hline
	
		We achieved our goals by distributing them across the team in the following manner:
		\begin{itemize}
			\item Web Server:
				\begin{itemize}
					\item Create Alexa skills schema: Xiaobin
					\item Alexa endpoint: {\color{red} ???}
					\item Endpoint for unlocking/locking bolt lock: Gwion
					\item Become an OAuth provider: Gwion, Ben
				\end{itemize}
			\item Android:
				\begin{itemize}
					\item UI \& server integration for unlocking/locking a bolt lock: thanks to good software architecture, this did not involve making any changes to the app code.
					\item Using OAuth for authentication: Theo
					\item UI mock-up and user testing of this: Ben, Theo
				\end{itemize}
			\item Robot:
			\begin{itemize}
				\item 3D-printing: Luke
				\item Buttons for manual robot control: Spencer
				\item Bolt lock mechanism: {\color{red} ???}
			\end{itemize}
		\end{itemize}

		As before we are still using GitHub's \textit{Projects} feature to track our progress on the software side, along with communicating on Slack and in our weekly meetings.
		This is the current state of our progress according to the \textit{Projects} tracking feature, in which
		we can easily define \textit{cards} describing tasks to be done, as of 13/3/2019.
		
		\vspace{3mm}
		
		\begin{tabular}{| c || c | c | c |} \hline
			\textbf{Section} & \textbf{Remaining cards} & \textbf{Cards in progress} & \textbf{Finished cards}\\ \hline
			Robot & 1 & 1 & 5 \\
			Web server & 10 & 5 & 11 \\
			Android App & 5 & 1 & 19 \\ \hline
		\end{tabular}

		\vspace{3mm}

		Note that these \textit{cards} do not map directly to our demo goals; for example one of the web server cards is \textit{Create API specification for Smart Home Adapter clients},
		which doesn't lend itself to be targeted for a certain demo as the API specification will need to be changed as we add and modify functionality.
	
  \\
  \hline
\end{tabular}
\vskip 5mm

% ------------ACHIEVEMENTS----------

\begin{tabular}{|p{\colWidth}|}
	\hline
	\cellcolor{blue!25}\large
	\textbf{What were your main achievements?}
	\\ \hline
	\vtop to 95mm{
		\color{red} ???
	}
  \\
  \hline
\end{tabular}
\vskip 5mm

% ------------NOT ACHIEVED----------

\begin{tabular}{|p{\colWidth}|}
	\hline
	\cellcolor{blue!25}\large
	\textbf{What did you not achieve? Briefly explain why.}
	\\ \hline
	\vtop to 95mm{
		\color{red} ???
  }
  \\
  \hline
\end{tabular}
\vskip 5mm

% ------------QUANTITIVE----------

\begin{tabular}{|p{\colWidth}|}
	\hline
	\cellcolor{blue!25}\large
	\textbf{Include any quantitative data you have collected (this can be a graph/table with a few words)}
	\\ \hline
	\vtop to 135mm{
		\color{red} Power draw? UI testing and vulnerability analysis isn't \textit{quantitative}

  }
  \\
  \hline
\end{tabular}
\vskip 5mm

% ------------NEXT STEPS----------

\begin{tabular}{|p{\colWidth}|}
	\hline
	\cellcolor{blue!25}\large
	\textbf{Say briefly what changes you will make to your plan for the next demo.}
	\\ \hline
	\vtop to 45mm{
		After internal discussion we have realised that we may have to cut the \textit{Triggers} feature,
		i.e. the user being able to set up rules such as "raise the thermostat by 2 degrees if the temperature ever drops below 20",
		or "turn the lights on when the sun sets".
		This is because we want to put a large focus on the marketability of our product for the final demo, which we believe sets
		our project apart from many others, and in order to do this we have chosen to polish our designs as much as possible.
		When we first set out we did not intend to have moved away from LEGO for all use cases nor did we originally plan on designing our own PCB, however these are things we believe will be more valuable to our project than the \textit{Triggers} function.
		Similarly, the time saved implementing this feature on the app and server sides can instead be redirected towards implementing the new UI to make for a better, more professional user experience.
		
		\vspace{3mm}
		
		{\color{red} Deep sleep? What to say?}
		
  	}
  \\
  \hline
\end{tabular}

\end{center}
  
\end{document}
