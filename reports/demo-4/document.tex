\documentclass[a4paper]{article}
\usepackage{array}  
\usepackage[table]{xcolor}% http://ctan.org/pkg/xcolor
\usepackage{geometry}
\geometry{margin=1.25in}
\usepackage{hhline}
\usepackage{environ}
 %\geometry{
 %a4paper,z
 %total={170mm,257mm},
 %left=40mm,
 %right=40mm
 %}
 \newcommand{\colWidth}{141mm}

\begin{document} 
\section*{Demo day: \textit{4} Group \textit{18}}

% ------------GOALS----------

\begin{center}
\begin{tabular}{|p{\colWidth}|}
	\hline
	\cellcolor{blue!25}\large
	\textbf{What were your goals?}
	\\ \hline
	
		For this demo, we focused our goals around polishing our existing systems.
		The following is a complete breakdown of what aspects we aimed at improving:
		\begin{itemize}
			\item Web server:
			\begin{itemize}
				\item Send push notifications to app when robot says it is low on battery
				\item New UI for OAuth web page
				\item Finish Alexa support
				\item Refactor existing systems to make them more easily extendable
				\item Make changes to accommodate new UI
				\begin{itemize}
					\item Add real names to user accounts (to welcome them with in the app)
					\item Endpoint to delete a robot from the user's account
					\item Endpoint to recalibrate a robot already registered to the user
					\item Endpoint to rename a robot already registered to the user
					\item Return robot statuses when fetching all robots
				\end{itemize}
				\item Penetration testing analysis
			\end{itemize}
			\item Android app:
			\begin{itemize}
				\item Receive push notifications from the server
				\item Implement new UI
				\begin{itemize}
					\item Move to single-main-page architecture
					\item Interact with robots directly from the main page
					\item New Robot Registration Wizard UI
					\item New Configuration Process \& UI as a response to last demo's usability testing
					\item New Login/Registration activity UI
				\end{itemize}
				\item Automatically deploy signed APKs to GitHub Releases when a commit is pushed into the master branch
				\item Add support for some non-English languages/locales
				\begin{itemize}
					\item Chinese
					\item Swedish
					\item Japanese
					\item German
					\item Hungarian
					\item Polish
					\item Portuguese
				\end{itemize}
			\end{itemize}
			\item Robot:
			\begin{itemize}
				\item Send request to server when battery is low
				\item Prepare set-up for final demonstration
				\item 3D-printed HAL design
				\item Final 3D-printed mechanisms for Switch, Thermostat, and Boltlock PALs
				\item Complete PCB design
				\item Light sleep for the ESP, to reduce power draw
				\item Final power consumption analysis
			\end{itemize}
		\end{itemize}
  \\
  \hline
\end{tabular}
\vskip 5mm

% ------------ORGANISATION----------

\begin{tabular}{|p{\colWidth}|}
	\hline
	\cellcolor{blue!25}\large
	\textbf{Summarise how your group organised the workload to achieve your goals.}
	\\ \hline
	\vtop to 120mm{
		Similarly to before, we largely allocated tasks according to the following structure:
		\begin{itemize}
			\item Web server:
			\begin{itemize}
				\item \textit{New UI for OAuth web page}: Ben
				\item \textit{Alexa Support}: Xiaobin
				\item \textit{Refactor existing systems to make them more easily extendable}: Ben, Gwion
				\item \textit{Make changes to accommodate new UI}: Gwion, Ben
				\item \textit{Penetration testing analysis}: Gwion
			\end{itemize}
			\item Android app:
			\begin{itemize}
				\item \textit{Implementation of the new UI}: Theo
				\item \textit{Design of the new UI}: Ben
				\item \textit{Translations}: Han, Theo, and various friends of Spencer's
			\end{itemize}
			\item Robot:
			\begin{itemize}
				\item \textit{Prepare set-up for final demonstration}: Han
				\item \textit{3D-printing}: Luke
				\item \textit{Thermostat PAL}: Han
				\item \textit{Complete PCB design}: Spencer
				\item \textit{Light sleep for the ESP, to reduce power draw}: Xiaobin, Spencer
				\item \textit{Final power consumption analysis}: Spencer, Xiaobin
			\end{itemize}
		\end{itemize}

		We've also continued to make heavy use of our Slack channel and GitHub Projects to keep track of
		everyone's progress. However, for this demo we made an increased effort to work side-by-side on site, which we felt was
		necessary as we near the end of the project.
  }
  \\
  \hline
\end{tabular}
\vskip 5mm

% ------------ACHIEVEMENTS----------

\begin{tabular}{|p{\colWidth}|}
	\hline
	\cellcolor{blue!25}\large
	\textbf{What were your main achievements?}
	\\ \hline
	\vtop to 95mm{
		We managed to achieve almost all (see next section) of our goals for the web server and the Android app.
		This means that we now provide a much more elegant and professional user experience,
		largely supporting 8 different languages.

  }
  \\
  \hline
\end{tabular}
\vskip 5mm

% ------------NOT ACHIEVED----------

\begin{tabular}{|p{\colWidth}|}
	\hline
	\cellcolor{blue!25}\large
	\textbf{What did you not achieve? Briefly explain why.}
	\\ \hline
	\vtop to 95mm{
		We did not achieve our goals related to low battery detection / sending push notifications. This is because the PCB
		was finished way later than we expected, partly due to it being a significant technical challenge and partly due to some
		unfortunately delayed deliveries (looking at you Mouser). We don't feel this impacts our ability to show off our product
		for this demo and are confident that we can get it working for the investor's fair on Friday.
  }
  \\
  \hline
\end{tabular}
\vskip 5mm

% ------------QUANTITIVE----------

\begin{tabular}{|p{\colWidth}|}
	\hline
	\cellcolor{blue!25}\large
	\textbf{Include any quantitative data you have collected (this can be a graph/table with a few words)}
	\\ \hline
	\vtop to 135mm{

	\textbf{PCB power consumption analysis}. These tests confirm our suspisions that a 1000mAh battery will not be enough for smooth usage unless
	we are able to change the way the devices are woken up from sleep mode so that they can be put into deep sleep.

	\vspace{3mm}

	\begin{tabular}{| c | c |} \hline
		Average power draw over 10 minutes & 12mA \\
		Expected lifetime with 1000mAh battery & 3.5 days \\
		Expected lifetime with 3000mAh battery & 10.5 days \\ \hline
	\end{tabular}

	\vspace{3mm}

	\textbf{Penetration testing of the webserver}.
	Smart home and Internet of Things (IoT) device security is of utmost importance as users' personal lives,
	privacy and homes may be at risk. As such, testing focused on identifying actual risks and vulnerabilities
	which may impact the users of our services.

	\vspace{3mm}

	\begin{tabular}{| c | c | c |} \hline
		\textbf{Identified issue} & \textbf{Risks involved} & \textbf{Severity} \\ \hline
		Unlock another user's boltlock & Burglary or worse & High \\
		Control another user's thermostat & Annoyance and wasted energy/money & Medium \\
		Control another user's light system & Annoyance and dissatisfied customers & Medium \\ \hline
	\end{tabular}

	\vspace{3mm}

	In short, the penetration testing exposed a vulnerability that allowed the attacker to impersonate
	any user, gaining them complete control over all of their registered devices. The consultant recommended
	the following to improve the security of our systems:

	\begin{itemize}
		\item Keep a record of which services and ports are accessible from outside networks
		\item Limit access to internal services; any service which does not need to be accessible from outside networks should not be
		\item In the longer term, consider a protection such as 2-factor authentication
	\end{itemize}

	The full report can be found attached to the back of this handout.
  }
  \\
  \hline
\end{tabular}
\vskip 5mm

% ------------NEXT STEPS----------

\begin{tabular}{|p{\colWidth}|}
	\hline
	\cellcolor{blue!25}\large
	\textbf{Say briefly what changes you will make to your plan for the next demo.}
	\\ \hline
	\vtop to 45mm{
		Though this is the last demo of the course, we do still cards up our sleeve for Friday:

		\begin{itemize}
			\item Low battery detection \& push notification sending
			\item Re-do power draw analysis one final time (due to a shortage of inductors we had to run the tests on serial-to-usb power instead of battery power)
			\item Prepare flyers and posters
		\end{itemize}
  }
  \\
  \hline
\end{tabular}

\end{center}
  
\end{document}